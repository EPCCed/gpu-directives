\documentclass{article}

\begin{document}

\title{GPU Programming with Directives Exercise \\\
  Logging on, Compiling and Running}

\author{}
\date{}

\maketitle

\section{Introduction}

The purpose of this exercise is to check you can compile and run
simple GPU programs on ARCHER2.

\section{Connecting to ARCHER2}

Log on to ARCHER2:
\begin{verbatim}
  ssh username@login.archer2.ac.uk
\end{verbatim}

\noindent Note: for this exercise, it is important to use the \texttt{/work}
file system and not the \texttt{/home} file system:

\begin{verbatim}
  cd /work/project/project/username/
\end{verbatim}

\section{Compiling and Running}
The code is contained in \texttt{saxyp.tar}. The tar file can be fetched from GitHub by
cloning the course repository with the following commands:

\begin{verbatim}
  git clone https://github.com/EPCCed/archer2-GPU-directives.git
  cd archer2-GPU-directives
  git checkout 2025-04-22
  cd Exercises
\end{verbatim}

\noindent Alternatively, the file can be found on ARCHER2 and copied into your
\texttt{/work} directory with the command:

\begin{verbatim}
  cp /work/z19/shared/GPUdir/saxpy.tar .
\end{verbatim}

\noindent Now unpack the file:
\begin{verbatim}
  tar -xvf saxpy.tar
\end{verbatim}

\section{Compiling and Running}

Since this exercise will involve offloading to GPUs with OpenMP
directives, certain modules must be loaded prior to compiling the
code:

\begin{verbatim}
  module load PrgEnv-amd
  module load rocm
  module load craype-accel-amd-gfx90a
  module load craype-x86-milan
\end{verbatim}

\noindent There are two examples, one using HIP and the other OpenMP. To compile and run the HIP example:

\begin{verbatim}
  cd saxpy
  make -f Makefile-hip
  sbatch hipsaxpy.slurm
\end{verbatim}

\noindent and similarly for the OpenMP example.

\end{document}
